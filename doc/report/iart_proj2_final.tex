% Copyright (C) 2021 Diogo Rodrigues, Rafael Ribeiro, Bernardo Ferreira
% Distributed under the terms of the GNU General Public License, version 3

\documentclass{beamer}
% Encodings (to render letters with diacritics and special characters)
\usepackage[utf8]{inputenc}
% Language
\usepackage[english]{babel}
\usepackage{verbatim}

\usepackage{multicol}

\usetheme{Madrid}
\usecolortheme{default}

\pdfstringdefDisableCommands{
  \def\\{}
  \def\texttt#1{<#1>}
}

\newcommand{\email}[1]{
{\footnotesize \texttt{\href{mailto:#1}{#1}} }
}

\usepackage{caption}
\DeclareCaptionFont{black}{\color{black}}
\DeclareCaptionFormat{listing}{{\tiny \textbf{#1}#2#3}}
\captionsetup[lstlisting]{format=listing,labelfont=black,textfont=black}

\usepackage{listings}
\lstset{
    frame=tb, % draw frame at top and bottom of the code
    basewidth  = {0.5em,0.5em},
    numbers=left, % display line numbers on the left
    showstringspaces=false, % don't mark spaces in strings  
    commentstyle=\color{green}, % comment color
    keywordstyle=\color{blue}, % keyword color
    stringstyle=\color{red}, % string color
	aboveskip=-0.2em,
    belowskip=-0.2em,
    basicstyle=\tiny
}
\lstset{literate=
  {á}{{\'a}}1 {é}{{\'e}}1 {í}{{\'i}}1 {ó}{{\'o}}1 {ú}{{\'u}}1
  {Á}{{\'A}}1 {É}{{\'E}}1 {Í}{{\'I}}1 {Ó}{{\'O}}1 {Ú}{{\'U}}1
  {à}{{\`a}}1 {è}{{\`e}}1 {ì}{{\`i}}1 {ò}{{\`o}}1 {ù}{{\`u}}1
  {À}{{\`A}}1 {È}{{\'E}}1 {Ì}{{\`I}}1 {Ò}{{\`O}}1 {Ù}{{\`U}}1
  {ä}{{\"a}}1 {ë}{{\"e}}1 {ï}{{\"i}}1 {ö}{{\"o}}1 {ü}{{\"u}}1
  {Ä}{{\"A}}1 {Ë}{{\"E}}1 {Ï}{{\"I}}1 {Ö}{{\"O}}1 {Ü}{{\"U}}1
  {â}{{\^a}}1 {ê}{{\^e}}1 {î}{{\^i}}1 {ô}{{\^o}}1 {û}{{\^u}}1
  {Â}{{\^A}}1 {Ê}{{\^E}}1 {Î}{{\^I}}1 {Ô}{{\^O}}1 {Û}{{\^U}}1
  {Ã}{{\~A}}1 {ã}{{\~a}}1 {Õ}{{\~O}}1 {õ}{{\~o}}1
  {œ}{{\oe}}1 {Œ}{{\OE}}1 {æ}{{\ae}}1 {Æ}{{\AE}}1 {ß}{{\ss}}1
  {ű}{{\H{u}}}1 {Ű}{{\H{U}}}1 {ő}{{\H{o}}}1 {Ő}{{\H{O}}}1
  {ç}{{\c c}}1 {Ç}{{\c C}}1 {ø}{{\o}}1 {å}{{\r a}}1 {Å}{{\r A}}1
  {€}{{\euro}}1 {£}{{\pounds}}1 {«}{{\guillemotleft}}1
  {»}{{\guillemotright}}1 {ñ}{{\~n}}1 {Ñ}{{\~N}}1 {¿}{{?`}}1
}

\usepackage{dirtree}

\usepackage[style=british]{csquotes}

\usepackage{tabularx}

\usepackage{graphicx}
	\graphicspath{{./images/}{../documentacao/}}
 
%Information to be included in the title page:
\AtBeginDocument{
\title[Ball Sort Puzzle - RL (Final delivery)]{Ball Sort Puzzle -- Reinforcement Learning}
\subtitle[]{Final delivery}
\author[Group 48]{
\begin{tabular}{r l}
	\email{up201806330@fe.up.pt} & Rafael Soares Ribeiro               \\
	\email{up201806429@fe.up.pt} & Diogo Miguel Ferreira Rodrigues     \\
	\email{up201806581@fe.up.pt} & Bernardo António Magalhães Ferreira
\end{tabular}
}
\institute[FEUP/IART]{Faculdade de Engenharia da Universidade do Porto \\ Artificial Intelligence (IART) -- Group 48}
\date[16/05/2021]{24th of May, 2021}
}

\begin{document}
\frame{\titlepage}

\begin{frame}
\frametitle{1. Work specification}
\framesubtitle{1.1. Problem description}

Solve solitaire game \href{https://play.google.com/store/apps/details?id=com.spicags.ballsort&hl=pt_PT&gl=US}{\textit{Ball Sort Puzzle}} by \href{https://play.google.com/store/apps/developer?id=Spica+Game+Studio}{Spica Game Studio} heuristic search methods.

Starting with a set of differently coloured balls distributed at random in different tubes, sort them so each tube has balls of a single color.

\vspace{0.5em}

\begin{minipage}{0.42\textwidth}
  \begin{itemize}
    \itemsep0em
    \item There are more tubes than colors;
    \item There are as many balls of a color as can fit in a tube;
    \item Cannot place more balls in a tube than it can hold;
    \item Can only move a ball on top of same-color ball (or tube is empty).
  \end{itemize}
\end{minipage}%
\begin{minipage}{0.58\textwidth}
  \centering
  \includegraphics[width=29mm]{img/lvl6-begin.png}
  \includegraphics[width=29mm]{img/lvl6-end.png}
\end{minipage}

\end{frame}

\begin{frame}
\frametitle{1. Work specification}
\framesubtitle{1.2. Problem formalization}

\textbf{State} | Vector $S = \langle t_1, t_2, ..., t_N \rangle$ of $N$ tubes with maximum capacity $H$, where tube $i$ is a stack $t_i = \langle t_i(1), t_i(2), ..., t_i(h_i) \rangle$ with $h_i$ balls ($0 \leq h_i \leq H$), where $t_i(1)$ is the color of the ball at the bottom, and $t_i(h_i)$ the color at the top. There are $C$ colors ($C < N$) numbered sequentially, thus $\forall i, j,~1 \leq t_i(j) \leq C$.

\textbf{Action} | The only action is a move:
\begin{itemize}
  \item Given state $S$, choose origin \& destination tubes $1 \leq i, j \leq N$ ($i \neq j$).
  \item Pre-conditions: $h_i \geq 1$, $h_j < H$ and either $h_j = 0$ or $t_i(h_i)=t_j(h_j)$.
  \item Cost: $1$ (one).
  \item Post-conditions: $S'$ is built from $S$, except $t_i$ is a ball shorter (say it had color $c$), and $t_j$ has one more ball at the top with color $c$.
\end{itemize}

\textbf{Policy} | This is what we want to find.

\end{frame}

\begin{frame}

\textbf{Reward} | We first tried the following reward function:

\begin{equation*}
  r = \begin{cases}
    WinReward & \text{if wins} \\
    LoseReward & \text{if loses} \\
    MoveHitReward & \text{if chosen move is valid} \\
    MoveMissReward & \text{if chosen move is invalid}
  \end{cases}
\end{equation*}

Sample values:
\begin{center}
  \footnotesize
  \begin{tabular}{c | c}
    $WinReward = +1$ & $LoseReward=-1$ \\ \hline
    $MoveHitReward = +0.01$ & $MoveMissReward = -0.01$
  \end{tabular}
\end{center}

Then we tried a new reward function:

\begin{equation*}
  r = \begin{cases}
    1 - \frac{1}{4}\frac{nMisses}{MaxMoves} - \frac{1}{4} \frac{nMoves}{MaxMoves} & \text{if wins} \\
    -1 & \text{if loses} \\
    0 & \text{otherwise}
  \end{cases}
\end{equation*}

($MaxMoves = 5000$)

\end{frame}

\begin{frame}
\frametitle{1. Work specification}
\framesubtitle{1.3. Solution description}

\begin{enumerate}
  \itemsep0em
  \item Implement the game
  \item Implement a human-friendly interface.
  \item Use machine learning algorithms to solve the game, and compare performances.
\end{enumerate}

\end{frame}

\begin{frame}
\frametitle{2. Related works}

A few common problems related to stacks, mostly applied to cargo containers.

Assume there is a set of cargo containers in a yard with a certain number of slots, and that containers are stacked to save space.

\vspace{0.5em}

\begin{tabular}{@{}p{46mm} p{70mm}@{}}
  \textbf{Container Pre-Marshalling Problem (CPMP)} & \textbf{Block Relocation Problem (BRP)} \\
  Some containers have more priority than others (e.g., are meant to be shipped first than others), so each stack must be sorted in non-decreasing order of priority from the bottom up with the least number of moves. &
  Containers will be extracted in a specific order (say that the $N$ containers are numbered from $1$ to $N$; container 1 is extracted first, then 2, ..., and finally container $N$). The goal is to extract all containers in that order with the least amount of moves (i.e., if you need to remove container 3 to extract container 1, you'd prefer to put 3 on top of 4 than on top of 2, otherwise you'd have to again relocate 3 to reach 2).
\end{tabular}

\end{frame}

% \begin{itemize}
% \item In both problems, search methods such as A* (with carefully considered heuristics) and branch-and-bound are regularly applied and yield good results, although optimal methods are clearly only reasonable in small scenarios, or to evaluate quality of sub-optimal, faster heuristic search methods.

% \item \cite{bortfeldt2012} provided a lower bound for the number of moves (very important to use as admissible heuristic for A*), and applied a tree search procedure to CPMP. \cite{tierney2017} further applied branching/symmetry breaking rules and iterative deepening A* to increase the size of problems that can be solved optimally.

% \item \cite{tricoire2018} proposed heuristics/metaheuristics for large BRP instances, and compared those approaches to branch-and-bound algorithms.
% \end{itemize}

\begin{frame}

\frametitle{2. Related works}
\framesubtitle{2.1. Combinatorial search problems}

There are many approaches to combinatorial search problems \cite{kotthoff2014}, but we are yet to fully explore this article and the methods it surveys.

\end{frame}

\begin{frame}

\frametitle{2. Related works}
\framesubtitle{2.1. Container Pre-Marshalling Problem (CPMP)}

\begin{itemize}
  \item Has been approached before with machine learning:
  \begin{itemize}
    \item \textbf{Reinforcement learning:} Q-learning \cite{hirashima2008}
    \item \textbf{Supervised learning:} \cite{hottung2020} labels nodes using known near-optimal algorithms, and uses a tree search together with deep neural networks (DNN) to select branches and prune the search tree, claims to achieve results 2\% away from optimality.
  \end{itemize}
  % \item Many mention classical approaches focus on search methods and strict lower bounds developed by experts in optimization and in the specific problem \cite{hottung2020, jovanovic2015, tierney2015}, and as such some prefer to focus on the \textbf{heuristics} to use:
  % \begin{itemize}
  %   \item There are plenty of heuristics to use \cite{jovanovic2015}.
  %   \item (Semi-automatic) \textbf{feature analysis} was also successfully applied \cite{tierney2015}
  % \end{itemize}
  % \item \textbf{Issues with greedy strategies} which are commonly applied can be overcome with ant-colony optimization \cite{jovanovic2015}, which shares similarities with \textbf{$\varepsilon$-greedy} methods.
\end{itemize}

\end{frame}

\begin{frame}

\frametitle{2. Related works}
\framesubtitle{2.1. Block Relocation Problem (BRP)}
\begin{itemize}
  \item Reinforcement learning has also been applied to BRP; \cite{jiang2021} provides a framework for reinforcement learning on BRP, and applies an heuristic function and $\varepsilon$-greedy strategy.
  % \item Many research has been applied to finding lower bounds, and \cite{zhang2020} presents innovative upper and lower bounds for search tree pruning
  % \item Optimal branch\&bound algorithms with heavy prunning have been applied with decent results for reasonably small instances, which can be improved upon by using a beam search algorithm (basically a best-first search with less memory usage).
\end{itemize}

This research is significant for our problem; even though the problems are quite different in nature, they share some traits with our problem, and can contribute with ideas for our own solutions.

\end{frame}

\begin{frame}
  \frametitle{3. Tools and algorithms}

  \href{https://github.com/Unity-Technologies/ml-agents}{\textbf{ML agents}} | Reinforcement learning
  \begin{itemize}
    \item PPO (Proximal Policy Optimization), on-policy "simple" first-order methods \cite{openai-ppo, schulman2017}
    \item SAC (Soft-Actor Critic), off-policy stochastic policy updating with results that are good and stable over different instances of the same problem \cite{openai-sac, haarnoja2018}
    \item POCA (Parallel Online Continuous Arcing), parallel boosting algorithm \cite{reichler2004}
  \end{itemize}
\end{frame}


\begin{frame}
\frametitle{4. Implementation}

\begin{itemize}
  \item \textbf{Language} | C\#
  \item \textbf{Environment} | Unity engine (v2020.2.2)
\end{itemize}

\textbf{Summary:} game and puzzle generation implemented, human interface implemented, trained PPO, SAC models

\begin{center}
  \includegraphics[width=75mm]{img/game-interface.png}
\end{center}

\end{frame}

\begin{frame}
\frametitle{5. Results}
\framesubtitle{5.1. PPO}

\textbf{GRÁFICOS DE VÁRIOS PPOs}

\textbf{Results of best PPO configurations}

Averaged over 10 random episodes

\begin{center}
  \begin{tabular}{l | r | r}
                          & \textbf{Old reward} & \textbf{New reward} \\ \hline
    Success rate (\%)     &                     & $100$ \\
    Number of (hit) moves &                     & $22.3$ \\
    Miss rate (\%)        &                     & $84.02$ \\
  \end{tabular}
\end{center}

\end{frame}

\begin{frame}
\frametitle{5. Results}
\framesubtitle{5.2. SAC}

\textbf{GRÁFICOS DE VÁRIOS SACs}

\textbf{Results of best SAC configurations}

\begin{center}
  \begin{tabular}{l | r | r}
                          & \textbf{Old reward} & \textbf{New reward} \\ \hline
    Success rate (\%)     &                     & \\
    Number of (hit) moves &                     & \\
    Miss rate (\%)        &                     & \\
  \end{tabular}
\end{center}

\end{frame}

\begin{frame}[allowframebreaks]
  \scriptsize
  \frametitle{Bibliography}
  \setbeamertemplate{bibliography item}{\insertbiblabel}
  \bibliographystyle{acm}
  \bibliography{report}
\end{frame}

\end{document}
